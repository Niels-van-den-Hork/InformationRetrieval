\documentclass[12pt]{article}

\usepackage{a4wide} % Hierdoor worden de margins van de pagina kleiner
\usepackage{graphicx} % \includegraphics[scale=0.8]{}
\usepackage{enumerate} % Voor mooiere opsommingen
\usepackage{amsmath} %basic math: https://en.wikibooks.org/wiki/LaTeX/Mathematics
\usepackage{mathtools}% more math: https://en.wikibooks.org/wiki/LaTeX/Advanced_Mathematics
\usepackage{wrapfig} %Better text around pictures \begin{wrapfigure}{r/R/l/L/o/O/i/I}{0.5\textwidth} https://en.wikibooks.org/wiki/LaTeX/Advanced_Mathematics
\usepackage{listings} %\begin{lstlisting} create code block https://en.wikibooks.org/wiki/LaTeX/Source_Code_Listings
\usepackage[normalem]{ulem} %striketrough \sout{} 
\usepackage{multicol} %tabel
\usepackage{multirow} %tabel
\usepackage{hyperref} %referenties in inhoudsopgave
%\usepackage {tikz} %graphs
%\usepackage[margin=2cm]{geometry}
%\usepackage{lstautogobble} ?

%\usepackage[nodoi]{apacite} 
%\bibliographystyle{apacite}
%\renewcommand\bibname{bibliografie}

\DeclarePairedDelimiter{\floor}{\lfloor}{\rfloor} %floor func
\DeclarePairedDelimiter{\ceil}{\lceil}{\rceil} %ceil func

\setlength\parindent{0pt} % noindent
%\setlength{\parskip}{0.1em}
%\setlength{\columnsep}{1cm}

\newcommand{\ifff}{\quad\leftrightarrow\quad} %if and only if
\newcommand{\sep}{ \quad;\quad}
\newcommand*\xor{\mathbin{\oplus}}
\newcommand{\inv}[1]{\frac{1}{#1}}
\newcommand{\img}[1]{\includegraphics[scale=0.6]{#1}}
\newcommand{\writeline}{\\ \\  \\ \line}
\newcommand{\citaA}[1]{\citeA{#1}}
\newcommand{\cita}[1]{\cite{#1}}

\renewcommand{\line}{\noindent\rule{16cm}{0.1pt}}
\renewcommand{\deg}{^{\circ}}
\renewcommand{\baselinestretch}{0.9}
\renewcommand{\mod}[1]{ $ (mod #1) $ }

\newcommand{\sect}[1]{\section{#1} \label{#1} }
\newcommand{\ssect}[1]{\subsection{#1} \label{#1} }
\newcommand{\sssect}[1]{\subsubsection{#1} \label{#1} }
\newcommand{\ssssect}[1]{\subsubsection*{#1} \label{#1} }
\newcommand{\ttref}[1]{\ref{#1} \textit{#1}}

\begin{document}

\title{Information Retrieval}
\author{
Manuel Minguez Carretero (s1234567), Martijn Schilpzand (s1234567),\\ Niels van den Hork (s4572602)}
\date{\today}


\maketitle



\sect{    Introduction / motivation}
We chose assignment 4 - DBPedia
We will transform queries in the dataset into queries that get better relevance results  when queried over nordlys \footnote{http://nordlys.cc/}. Nordlys will be our search engine, DBPedia\footnote{https://wiki.dbpedia.org/about} will be searched. DBPedia-entity \footnote{https://github.com/iai-group/DBpedia-Entity} will contain the queries and results including relevance which we will use to evaulate our transformation 

\sect{    Research question}
%How effective is Nordlys in retrieving entities?
Retrieving better entities when transform queries 
at first we will focus on *** type of query
we would for example transform 'How to cook pasta' to 'Pasta Recipe'


\sect{    Resources (data/software/users)}
We will be using the DBPedia Entity V2 collection

\sect{    Experimental design}


We will treat any entity that has no relevancy data is if it is not relevant
However, we can expect cases where those entities turn out to actually be relevant results to a query. With the recall value we can focus on ' did we get everything marked relevant'  

\sect{    Expected outcome}





\end{document}
